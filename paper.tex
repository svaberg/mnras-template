 % !TeX root = ./paper.tex
% !BIB program = bibtex

% chkTeX linting rules to ignore:
% chktex-file 2   Non-breaking space should have been used.
% chktex-file 3   You should enclose the previous parenthesis with {}
% chktex-file 8   Wrong length of dash may have been used
% chktex-file 21  This command might not be intended
% chktex-file 39  Double space found

% mnras_template.tex
%
% LaTeX template for creating an MNRAS paper
%
% v3.0 released 14 May 2015
% (version numbers match those of mnras.cls)
% Copyright (C) Royal Astronomical Society 2015
% Authors:
% Keith T. Smith (Royal Astronomical Society)

% Change log

%
% v3.0 May 2015
%    Renamed to match the new package name
%    Version number matches mnras.cls
%    A few minor tweaks to wording
% v1.0 September 2013
%    Beta testing only - never publicly released
%    First version: a simple (ish) template for creating an MNRAS paper

%%%%%%%%%%%%%%%%%%%%%%%%%%%%%%%%%%%%%%%%%%%%%%%%%%
% Basic setup. Most papers should leave these options alone.
\documentclass[a4paper,fleqn,usenatbib]{mnras}
%\documentclass[draft,a4paper,fleqn,usenatbib]{mnras}

% MNRAS is set in Times font. If you don't have this installed (most LaTeX
% installations will be fine) or prefer the old Computer Modern fonts, comment
% out the following line
\usepackage{newtxtext}
\usepackage{newtxmath}
\usepackage{dblfloatfix}  % To be removed
% Depending on your LaTeX fonts installation, you might get better results with one of these:
%\usepackage{mathptmx}
%\usepackage{txfonts}

% \usepackage{tensor}
% \usepackage{tensind}
% \tensordelimiter{?}

% Use vector fonts, so it zooms properly in on-screen viewing software
% Don't change these lines unless you know what you are doing
\usepackage[T1]{fontenc}
\usepackage{ae,aecompl}
% \usepackage{placeins}  % Only for the appendix where we use \FloatBarrier


%%%%% AUTHORS - PLACE YOUR OWN PACKAGES HERE %%%%%

%
% Only include extra packages if you really need them. Common packages are:
%
\usepackage{graphicx}	% Including figure files

% There are redefined below on my machine; the following
% code makes this not produce an error which stops the compilation.
\ifdefined\Bbbk{}
    \let\Bbbk\relax
\else
    % Do nothing.
\fi
\ifdefined\bigtimes{}
    \let\bigtimes\relax
\else
    % Do nothing.
\fi
% \usepackage{amsmath}	% Advanced maths commands (amsmath should be loaded by newtxmath)
% \usepackage{amssymb}	% Extra maths symbols (amssymb should be loaded by newtxmath)
\usepackage[separate-uncertainty=true,
            multi-part-units=single]{siunitx}
\usepackage{pgf}
\usepackage{xprintlen}
\usepackage{booktabs}

\let\widebar\relax  % See https://tex.stackexchange.com/questions/426091/command-widering-already-defined
\usepackage{mathabx}  % For Earth and Sun symbols (required \let\widebar\relax)

\usepackage{wasysym}

\usepackage{sparklines}  % No longer used
\usepackage{orcidlink} % This is for including the ORCID ID
\usepackage{lipsum}  % To generate dummy text


%
% Add pdf properties test.
%
\pdfminorversion=5  % Recommended in MNRAS submission

% \hypersetup{pdfusetitle=true}  % Could not get this to work
\hypersetup{pdfauthor={D. Evensberget et al.}}  % Might not work with mnras??
%%%%%%%%%%%%%%%%%%%%%%%%%%%%%%%%%%%%%%%%%%%%%%%%%%


%%%%% AUTHORS - PLACE YOUR OWN COMMANDS HERE %%%%%

% Please keep new commands to a minimum, and use \newcommand not \def to avoid
% overwriting existing commands. Example:

% TODO use \renewcommand
\newcommand{\batsrus}{{\textsc{bats-r-us}}}
\newcommand{\swmf}{{\textsc{swmf}}}
\newcommand{\awsom}{{\textsc{awsom}}}
\renewcommand{\pluto}{{\textsc{pluto}}} % Interferes with Pluto symbol in wasysym
\newcommand{\toupies}{TOUPIES}
\newcommand{\bcool}{{\textsc{BCool}}}
\newcommand{\ksi}{{\(\xi\)}}
\newcommand{\SF}{{\Phi}}  % To be able to call Phi0 just phi

% \definecolor{darkgreen}{rgb}{0.0, 0.5, 0.0}
% \newcommand{\summary}[1]{}
% \newcommand{\selfplagiate}[1]{}
% \newcommand{\todo}[1]{}
% \newcommand{\notes}[1]{}
% \newcommand{\review}[2]{}
% \newcommand{\response}[2]{#2}
% \renewcommand{\summary}[1]{{\color{darkgreen}#1}}  % If commented out, this hides all the summary fields
% \renewcommand{\selfplagiate}[1]{{\color{gray}#1}}  % If commented out, this hides all the self plagiate material to be rewritten.
% \renewcommand{\todo}[1]{{\color{magenta}{\bf TODO}: #1}}  % If commented out, this hides all the todo fields
% \renewcommand{\notes}[1]{{\color{magenta}#1}}  % If commented out, this hides the notes fields
% \renewcommand{\review}[2]{{\color{orange}\textsuperscript{(Reviewer comment #1)}\ignorespaces{}#2}}  % If commented out, this hides all the review formatting
% \renewcommand{\response}[2]{{\color{red}\textsuperscript{(#1)}\ignorespaces{}#2}}  % If commented out, this hides all the review formatting


% \newcommand{\tensorindices}[1]{{#1}}
% \renewcommand{\tensorindices}[1]{\mathrm{\emph{#1}}}
% \newcommand{\TI}[1]{\tensorindices{#1}}


% \newcommand{\stackpm}[2]{\substack{+#1 \\ -#2}}
        % \addlinespace[.5ex] Mel25-21  & \(0.90\substack{+0.05 \\ -0.05}\) & 0\\
        % \addlinespace[.5ex] Mel25-43  & \(0.90{\tiny +0.05 \atop -0.05}\) & 0\\

\listfiles
\newcommand{\Wind}{\textsc{w}} % "Wind" subscript
\newcommand{\Rot}{\text{rot}}  % "Rotation" subscript
\newcommand{\Obs}{\text{obs}}  % Observation-based gyrotrack (Gallet & Bouvier 2013)
\newcommand{\Sat}{\text{sat}}  % "Saturation" subscript
\newcommand{\Xray}{\textsc{x}} % X-ray subscript
\newcommand{\Alfven}{{\textsc{a}}} % Alfvén (radius, speed, etc.)
\newcommand{\corona}{\text{c}} % Corona subscript
\newcommand{\bol}{\text{bol}} % Bolometric subscript
\newcommand{\Mag}{\text{mag}} % "mag" magnetic subscript
\newcommand{\Open}{\text{open}} % "open" subscript
\newcommand{\HK}{\textsc{hk}} % Ca II H & K lines
\newcommand{\Star}{\text{\bigstar}} % Star subscript a la Sun subscript. A bit bigger than \star.
\newcommand{\CS}{{{\scriptscriptstyle{B}}}} % Current sheet subscript


% Citation aliases
% \defcitealias{2016MNRAS.457..580F}{F16}
% \defcitealias{2018MNRAS.474.4956F}{F18}
% Use like this: \citetalias{2016MNRAS.457..580F} and \citetalias{2018MNRAS.474.4956F}
% Or this: \citetalias{2016MNRAS.457..580F,2018MNRAS.474.4956F}

\defcitealias{2013A&A...556A..36G}{GB13}  % Gallet & Bouvier (2013)


%See https://tex.stackexchange.com/a/212071/120322 about \mathchoice
% TODO try with just \textsc{zdi}, etc.
\newcommand{\ZDI}{{\mathchoice{}{}{\scriptscriptstyle}{}\mathrm{ZDI}}}
\newcommand{\ZB}{{\mathchoice{}{}{\scriptscriptstyle}{}\mathrm{ZB}}}
\newcommand{\Hollweg}{{\mathchoice{}{}{\scriptscriptstyle}{}\mathrm{H}}}
\newcommand{\Spitzer}{{\mathchoice{}{}{\scriptscriptstyle}{}\mathrm{S}}}
\newcommand{\SStar}{{\mathchoice{}{}{\scriptscriptstyle}{}\mathrm{\bigstar}}}
\newcommand{\Skumanich}{\textsc{s}}

\renewcommand{\Earth}{{\mathchoice{}{}{\scriptscriptstyle}{}\oplus}} % Sun subscript
\renewcommand{\Sun}{{\mathchoice{}{}{\scriptscriptstyle}{}\odot}} % Sun subscript
\renewcommand{\Star}{{\mathchoice{}{}{\scriptscriptstyle}{}\bigstar}} % Sun subscript


\newcommand{\onlineref}[1]{\ref{#1} (on-line supplement)}
% \renewcommand{\onlineref}[1]{(on-line supplement)}


% SiUnitX aliases
\DeclareSIUnit\year{yr}
\DeclareSIUnit\astronomicalunit{au}
\DeclareSIUnit\parsec{pc}
\DeclareSIUnit\erg{erg}
\DeclareSIUnit\gauss{G}
\DeclareSIUnit\mSun{\mbox{\(M_\Sun\)}}
\DeclareSIUnit\rSun{\mbox{\(R_\Sun\)}}
\DeclareSIUnit\mEarth{\mbox{\(M_\Earth\)}}
\DeclareSIUnit\rEarth{\mbox{\(R_\Earth\)}}
% For the use of \mbox, see https://tex.stackexchange.com/q/239863

\renewcommand{\vec}[1]{\boldsymbol{#1}} % Boldface vectors instead of arrow superscript vectors
\newcommand{\uvec}[1]{\boldsymbol{\hat{#1}}} % Unit vector: Boldface with hat

\newcommand\thefont{\expandafter\string\the\font} % chktex 41

%Give proper spacing around \propto
%Found the underlying symbol by doing \(\show \propto\) and looking in the log.
\renewcommand{\propto}{\mathrel{\mathchar"939}} % chktex 18 this is if using newtxmath.
\newcommand{\mysim}{{\sim}} % Not so much spacing around for writing ~2 Gyr, etc.

% https://tex.stackexchange.com/questions/33538/how-to-get-an-approximately-proportional-to-symbol
\newcommand{\appropto}{\mathrel{\vcenter{  % chktex 41
  \offinterlineskip\halign{\hfil\(##\)\cr   % chktex 41
    \propto\cr\noalign{\kern2pt}\sim\cr\noalign{\kern-2pt}}}}}   % chktex 41

% Glyphs to identify cases
\usepackage{star-commands}


%%%%%%%%%%%%%%%%%%%%%%%%%%%%%%%%%%%%%%%%%%%%%%%%%%
%%%%%%%%%%%%%%%%%%% TITLE PAGE %%%%%%%%%%%%%%%%%%%
%
% Title of the paper, and the short title which is used in the headers.
% Keep the title short and informative.

\title[Rotational evolution]
    {Rotational evolution}
%
% The list of authors, and the short list which is used in the headers.
% If you need two or more lines of authors, add an extra line  using \newauthor
\author[D. Evensberget et al.]{
    D. Evensberget \orcidlink{0000-0001-7810-8028}\(^{1}\)\thanks{E-mail: \href{mailto:evensberget@strw.leidenuniv.nl}{evensberget@strw.leidenuniv.nl}},
    coauthors
    % A. A. Vidotto  \orcidlink{0000-0001-5371-2675}\(^{1}\),
    %,
    % K. M. Strickert  \orcidlink{0000-0001-6584-5969}\(^{1}\),
    % and
    % C. P. Folsom    \orcidlink{0000-0002-9023-7890}\(^{2,3}\)
    % R. D. Kavanagh  \orcidlink{0000-0002-1486-7188}\(^{5,1}\),
    % J. S. Pineda    \orcidlink{0000-0002-4489-0135}\(^{6}\),
    % \newauthor
    % F. A. Driessen  \orcidlink{0000-0003-3005-7377}\(^{1}\),
    % and maybe
    % S. C. Marsden  \orcidlink{0000-0001-5522-8887}\(^{2}\),
    % B. D. Carter   \orcidlink{0000-0003-0035-8769}\(^{2}\),
    % R. Salmeron    \orcidlink{0000-0002-1956-4493}\(^{2}\),
    \\
    % List of institutions
    \(^{1}\)Leiden Observatory, Leiden University, PO Box 9513, 2300 RA Leiden, The Netherlands
    % \\
    % \(^{2}\)Centre for Astrophysics, University of Southern Queensland, Toowoomba, Queensland 4350, Australia
    % \\
    % \(^{2}\)Tartu Observatory, University of Tartu, Observatooriumi 1, T\~{o}ravere, 61602 Tartumaa, Estonia
    % \\
    % \(^{3}\)University of Western Ontario, Department of Physics \& Astronomy, London, ON, N6A 3K7, Canada
    % \\
    % \(^{5}\)ASTRON, The Netherlands Institute for Radio Astronomy, Oude Hogeveensedijk 4, 7991PD, Dwingeloo, The Netherlands
    % \\
    % \(^{6}\)University of Colorado Boulder, Laboratory for Atmospheric and Space Physics, 3665 Discovery Drive, Boulder, CO 80303, USA
}

%
% These dates will be filled out by the publisher
\date{Accepted XXX.\@ Received YYY;\@ in original form ZZZ}
%
% Enter the current year, for the copyright statements etc.
\pubyear{2024}
%

% Don't change these lines
\begin{document}
\label{firstpage}  % chktex 24
\pagerange{\pageref{firstpage}--\pageref{lastpage}}
\maketitle
%
% Abstract of the paper
\begin{abstract}
    % chktex-file 46 Use \(...\) instead of $...$
Solar-type stars form with a wide range

\end{abstract}

% Select between one and six entries from the list of approved keywords.
% Don't make up new ones.
\begin{keywords}
    stars: winds, outflows --
    stars: rotation --
    stars: magnetic field --
    stars: solar-type --
    stars: evolution --
    Sun: evolution
\end{keywords}



%Figures are referred to as e.g.\ Fig.~\ref{fig:example_figure}, and tables as
%e.g.\ Table~\ref{tab:example_table}.


%%%%%%%%%%%%%%%%%%%%%%%%%%%%%%%%%%%%%%%%%%%%%%%%%%
% !TEX root = ./paper.tex


% chkTeX linting rules to ignore:
% chktex-file 2   Non-breaking space should have been used.
% chktex-file 3   You should enclose the previous parenthesis with {}
% chktex-file 8   Wrong length of dash may have been used
% chktex-file 21  This command might not be intended
% chktex-file 39  Double space found

%%%%%%%%%%%%%%%%% BODY OF PAPER %%%%%%%%%%%%%%%%%%

\section{Introduction}

In Section~\ref{sec:analysis}, we show this by comparing instantaneous wind torque \(\dot J_\Wind\) of our models to a simple observation-based gyrotrack model based on the work of
The resulting wind torques do however need to be scaled by a factor of \(\sim 7\) in order to yield s presented in Section~\ref{sec:Discussion}.
In Section~\ref{sec:Conclusions} we present our conclusions.



%
% SECTION
%
\section{Analysis}\label{sec:analysis}
% In this letter we use the subscripts
% \((\cdot)_\ZDI\),
% \((\cdot)_\Wind\), and
% \((\cdot)_\Obs\)
% to denote quantities from the surface magnetic maps, the numerical wind models, and \todo{what exactly}, respectively.

% \subsection{Three-dimensional wind models}
In our recent work we have modelled the winds of thirty



%
% SECTION
%
\section{Discussion}\label{sec:Discussion}

Multiple, possibly interacting physical phenomena

%
% SECTION
%
\section{Conclusions}\label{sec:Conclusions}
We have modelled the stellar spin-down rates

\section*{Acknowledgements}
% \begin{enumerate}
    % Funding
    % \item

    % \item
    This project has received funding from the European Research Council (ERC) under the European Union's Horizon 2020 research and innovation programme (grant agreement No 817540, ASTROFLOW).
    % \item
    This work used the Dutch national e-infrastructure with the support of the SURF Cooperative using grant nos. EINF-2218 and EINF-5173.
    % \item
    % We thank SURF (\url{www.surf.nl}) for the support in using the National Supercomputer Snellius.
    % \item
    This research has made use of NASA's \href{https://ui.adsabs.harvard.edu/}{Astrophysics Data System}.
    % Software
    % \item
    This work was carried out using the \swmf{} tools developed at The University of Michigan \href{https://spaceweather.engin.umich.edu/the-center-for-space-environment-modelling-csem/}{Center for Space Environment Modelling (CSEM)} and made available through the NASA \href{https://ccmc.gsfc.nasa.gov/}{Community Coordinated Modelling Center (CCMC)}.
    % \item
    This work has made use of the following additional numerical software, statistics software and visualisation software:
          NumPy~\citep{2011CSE....13b..22V},
          SciPy~\citep{2020SciPy-NMeth},
     Matplotlib~\citep{2007CSE.....9...90H},
     and
    statsmodels~\citep{seabold2010statsmodels}.
    % \item
    %We would like to express our gratitude to the anonymous reviewers for their valuable contributions, which have enhanced the quality and rigour of our article.
% \end{enumerate}

\section*{Data availability}
The data underlying this article will be shared on reasonable request to the corresponding author.

%%%%%%%%%%%%%%%%%%%%%%%%%%%%%%%%%%%%%%%%%%%%%%%%%%

%%%%%%%%%%%%%%%%%%%% REFERENCES %%%%%%%%%%%%%%%%%%

% The best way to enter references is to use BibTeX:
\bibliographystyle{mnras}
\bibliography{bibliography} % if your bibtex file is called bibliography.bib

%%%%%%%%%%%%%%%%%%%%%%%%%%%%%%%%%%%%%%%%%%%%%%%%%%

%%%%%%%%%%%%%%%%% APPENDICES %%%%%%%%%%%%%%%%%%%%%
\appendix
% !TEX root = ./paper.tex


% chkTeX linting rules to ignore:
% chktex-file 2   Non-breaking space should have been used.
% chktex-file 3   You should enclose the previous parenthesis with {}
% chktex-file 8   Wrong length of dash may have been used
% chktex-file 21  This command might not be intended
% chktex-file 39  Double space found

\section{Wind models}\label{sec:wind-models}
This work is based upon numerical modelling
%%%%%%%%%%%%%%%%%%%%%%%%%%%%%%%%%%%%%%%%%%%%%%%%%%


% Don't change these lines
\bsp	% typesetting comment % chktex 1
\label{lastpage} % chktex 24

\clearpage
% \normalsize  % Otherwise the font size is 8 pt instead of 9 pt in the appendix (for on-line supplement style separate from main paper)
\end{document}

% End of mnras_template.tex